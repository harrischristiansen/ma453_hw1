\documentclass[11pt]{article}
\input{headers1}

\usepackage{fancyhdr}   
\pagestyle{fancy}      
\lhead{MA453 Spring 2018 - Homework 1}               
\rhead{Harris Christiansen (christih@purdue.edu)}

\usepackage[strict]{changepage}  
\newcommand{\nextoddpage}{\checkoddpage\ifoddpage{\ \newpage\ \newpage}\else{\ \newpage}\fi}  


\begin{document}

\title{Homework 1}
\date{p24 B.5 and B.6, p29 A.1 and A.2, p39 A.1 and A.2, p40 C.3 and C.5}
\maketitle

\thispagestyle{fancy}  
\pagestyle{fancy}      

\begin{enumerate}

%%% Problem p24 B.5
\item {\bfseries p24 B.5.}
  $x*y = xy + 1$
  \begin{itemize}
  \item {\bfseries Commutative} Yes/No
  \item {\bfseries Associative} Yes/No
  \item {\bfseries Identity} Yes/No
  \item {\bfseries Inverses} Yes/No
  \end{itemize} 
 
%%% Problem p24 B.6
\item {\bfseries p24 B.6.}
  $x*y = \max\{x,y\}$ = the larger of the two numbers x and y
  \begin{itemize}
  \item {\bfseries Commutative} Yes/No
  \item {\bfseries Associative} Yes/No
  \item {\bfseries Identity} Yes/No
  \item {\bfseries Inverses} Yes/No
  \end{itemize} 
 
%%% Problem p29 A.1
\item {\bfseries p29 A.1.}
  Prove the following set is an abelian group:
  
  $x*y = x + y + k$ (k is a fixed constant), on the set $\mathbb{R}$ of the real numbers
  
  {\bfseries Solution.}
  Answer
 
%%% Problem p29 A.2
\item {\bfseries p29 A.2.}
  Prove the following set is an abelian group:
  
  $x*y = \frac{xy}{2}$ on the set $\{x \in \mathbb{R} : x \neq 0\}$
  
  {\bfseries Solution.}
  Answer
 
%%% Problem p39 A.1
\item {\bfseries p39 A.1.}
  Solve in terms of $a$, $b$, and $c$:
  
  $axb = c$
  
  {\bfseries Solution.}
  Answer
 
%%% Problem p39 A.2
\item {\bfseries p39 A.2.}
  Solve in terms of $a$, $b$, and $c$:
  
  $x^2b = xa^{-1}c$
  
  {\bfseries Solution.}
  Answer
 
%%% Problem p40 C.3
\item {\bfseries p40 C.3.}
  Assuming that $a$ and $b$ commute, prove the following:
  
  $a$ commutes with $ab$
  
  {\bfseries Solution.}
  Answer
 
%%% Problem p40 C.5
\item {\bfseries p40 C.5.}
  Assuming that $a$ and $b$ commute, prove the following:
  
  $xax^{-1}$ commutes with $xbx^{-1}$, for any $x \in G$
  
  {\bfseries Solution.}
  Answer


\end{enumerate}

\end{document}
