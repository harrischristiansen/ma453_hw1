\documentclass[11pt]{article}
\input{headers1}

\usepackage{fancyhdr}   
\pagestyle{fancy}      
\lhead{MA453 Spring 2018 - Homework 1}               
\rhead{Harris Christiansen (christih@purdue.edu)}

\usepackage[strict]{changepage}  
\newcommand{\nextoddpage}{\checkoddpage\ifoddpage{\ \newpage\ \newpage}\else{\ \newpage}\fi}  


\begin{document}

\title{Homework 1}
\date{p24 B.5 and B.6, p29 A.1 and A.2, p39 A.1 and A.2, p40 C.3 and C.5}
\maketitle

\thispagestyle{fancy}  
\pagestyle{fancy}      

\begin{enumerate}

%%% Problem p24 B.5
\item {\bfseries p24 B.5.}
  $x*y = xy + 1$
  \begin{itemize}
  \item {\bfseries Commutative} Yes
  \item {\bfseries Associative} Yes
  \item {\bfseries Identity} Yes
  \item {\bfseries Inverses} No
  \end{itemize} 
 
%%% Problem p24 B.6
\item {\bfseries p24 B.6.}
  $x*y = \max\{x,y\}$ = the larger of the two numbers x and y
  \begin{itemize}
  \item {\bfseries Commutative} Yes
  \item {\bfseries Associative} Yes
  \item {\bfseries Identity} Yes
  \item {\bfseries Inverses} No
  \end{itemize} 
 
%%% Problem p29 A.1
\item {\bfseries p29 A.1.}
  Prove the following set is an abelian group:
  
  $x*y = x + y + k$ (k is a fixed constant), on the set $\mathbb{R}$ of the real numbers
  
  {\bfseries Solution.}
  
  Let $x, y \in \mathbb{R}$ be real numbers. Forms in $*$ are {\bfseries commutative} iff $x*y = y*x$. \\
  $x*y = x + y + k$ \\
  $y*x = y + x + k$ \\
  $x + y + k = y + x + k$ \\
  Based upon the commutative nature of addition in the real numbers the above equations are equal $\qed$\\
  
  Let $x, y, z \in \mathbb{R}$ be real numbers. Forms in $*$ are {\bfseries associative} iff $(x*y)*z = y*(x*z)$. \\
  $(x*y)*z = (x + y + k) + z + k$ \\
  $x*(y*z) = x + (y + z +k) + k$ \\
  $x + y + k = y + x + k$ \\
  Based upon the commutative nature of addition in the real numbers the above equations are equal $\qed$\\
   
  Let $x,e \in \mathbb{R}$ be real numbers. Forms in $*$ have an {\bfseries identify} iff $x*e = x$. \\
  $x*e = x + e + k = x$ is true for all x with $e = -k$ $\qed$\\
  
  Let $x,x^{-1} \in \mathbb{R}$ be real numbers. Forms in $*$ have an {\bfseries inverse} iff $x*x^{-1} = 1$. \\
  $x*x^{-1} = x + x^{-1} + k = 1$ is true for all x with $x^{-1} = -x-k+1$ $\qed$\\
  
  Thus, because the group is true for commutative, associative, identify, and inverse, it is an abelian group. $\qed$\\
 
%%% Problem p29 A.2
\item {\bfseries p29 A.2.}
  Prove the following set is an abelian group:
  
  $x*y = \frac{xy}{2}$ on the set $\{x \in \mathbb{R} : x \neq 0\}$
  
  {\bfseries Solution.}
  
  Let $x, y \in \{x \in \mathbb{R} : x \neq 0\}$ be real numbers. Forms in $*$ are {\bfseries commutative} iff $x*y = y*x$. \\
  $x*y = \frac{xy}{2}$ \\
  $y*x = \frac{yx}{2}$ \\
  $\frac{xy}{2} = \frac{yx}{2}$ \\
  Based upon the commutative nature of multiplication in the real numbers the above equations are equal $\qed$\\
  
  Let $x, y, z \in \{x \in \mathbb{R} : x \neq 0\}$ be real numbers. Forms in $*$ are {\bfseries associative} iff $(x*y)*z = y*(x*z)$. \\
  $(x*y)*z = \frac{(\frac{xy}{2})z}{2}$ \\
  $x*(y*z) = \frac{x(\frac{yz}{2})}{2}$ \\
  $\frac{(\frac{xy}{2})z}{2} = \frac{x(\frac{yz}{2})}{2}$ \\
  Based upon the commutative nature of multiplication in the real numbers the above equations are equal $\qed$\\
   
  Let $x,e \in \{x \in \mathbb{R} : x \neq 0\}$ be real numbers. Forms in $*$ have an {\bfseries identify} iff $x*e = x$. \\
  $x*e = \frac{xe}{2} = x$ is true for all x with $e = 2$\\
  
  Let $x,x^{-1} \in \{x \in \mathbb{R} : x \neq 0\}$ be real numbers. Forms in $*$ have an {\bfseries inverse} iff $x*x^{-1} = 1$. \\
  $x*x^{-1} = \frac{xx^{-1}}{2} = 1$ is true for all x with $x^{-1} = 2/x$ $\qed$\\
  
  Thus, because the group is true for commutative, associative, identify, and inverse, it is an abelian group. $\qed$\\
 
%%% Problem p39 A.1
\item {\bfseries p39 A.1.}
  Solve in terms of $a$, $b$, and $c$:
  
  $axb = c$
  
  {\bfseries Solution.}\\
  $a = c/xb$\\
  $b = c/ax$\\
  $c = abx$\\
 
%%% Problem p39 A.2
\item {\bfseries p39 A.2.}
  Solve in terms of $a$, $b$, and $c$:
  
  $x^2b = xa^{-1}c$
  
  {\bfseries Solution.}\\
  $a = \frac{xc}{x^2b}$\\
  $b = \frac{xa^{-1}c}{x^2}$\\
  $c = \frac{x^2b}{xa^{-1}}$\\
 
%%% Problem p40 C.3
\item {\bfseries p40 C.3.}
  Assuming that $a$ and $b$ commute, prove the following:
  
  $a$ commutes with $ab$
  
  {\bfseries Solution.}
  Answer
 
%%% Problem p40 C.5
\item {\bfseries p40 C.5.}
  Assuming that $a$ and $b$ commute, prove the following:
  
  $xax^{-1}$ commutes with $xbx^{-1}$, for any $x \in G$
  
  {\bfseries Solution.}
  Answer


\end{enumerate}

\end{document}
